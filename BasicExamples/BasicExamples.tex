\documentclass{article}
% It is best to read a .tex file with a text editor that shows you line numbers.
% Any line in this document that starts with a % is a comment in the code and doesn't effect the final document.

%this package creates paragraphs filled with dummy text.
\usepackage{lipsum}

%sets the 4 margin sizes.
\usepackage[top=1in, bottom=1in, left=1.25in, right=1.75in]{geometry}

\author{David Goulette}
\date{\today}
\title{Basic Examples}

% Here is where the document actually begins.  Everything above this is just "set up stuff"
\begin{document}
\maketitle
\section{Enumerations and itemizations with default environments}
Here is an enumerated list with nested items. 
\begin{enumerate}
\item The first
\item The second
\begin{enumerate}
\item subitem one
\item subitem two
\begin{enumerate}
\item subsubitem
\end{enumerate}
\item subitem three
\end{enumerate}
\item The third
\end{enumerate}

Here is an itemized list with nested items.

\begin{itemize}
\item The first
\item The second
\begin{itemize}
\item subitem one
\item subitem two
\begin{itemize}
\item subsubitem
\begin{itemize}
\item subsubsubitem
\end{itemize}
\end{itemize}
\item subitem three
\end{itemize}
\item The third
\end{itemize}

\section{Center and flushright}
\begin{center}
This \texttt{text} is \textsc{centered}\\ in the\\ \textbf{middle}\\ of the\\ \textsl{page.}
\end{center}

\begin{flushright}
This text\\is\\flushed to\\the\\right.
\end{flushright}

\subsection{A subsection with dummy text using the \texttt{lipsum} package.}
\lipsum[1] %this prints the first paragraph of dummy text
\subsubsection{This is a sub-subsection with dummy text}
\lipsum[2] %this prints the second paragraph


{\large \underline{\textbf{\texttt{This is a forced page break}}}}.
\pagebreak

\section{New section on a new page.}
\lipsum[3-4]  %this prints the 3rd and 4th paragraph of dummy text
\end{document}